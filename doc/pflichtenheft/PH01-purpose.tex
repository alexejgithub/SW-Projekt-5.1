\section{Purpose}
The software project module in 2017 at the University of Constance focuses on the development of an app for mobile devices. \\
Especially, this Pflichtenheft intends to describe the structure of an implementation of a virtual reality representation of BLE sensor feedback.

\subsection{Product Idea and Goal}
The general goal of the app consists of two parts: \\
First, the user shall be able to record data with a sensor connected to the user`s smartphone via Bluetooth and save the data as a function of the respective location. \\
As the second part, the gathered data shall be displayed in a 3D environment which the user can explore in Virtual Reality. By displaying the data in regard to the location where it was recorded, it can be expierienced like a three-dimensional function where the user can navigate using a joystick. \\
If possible, it would be nice to be able to watch the graphs visualising the data change over a period of time. This would be like watching the data in a video while being able to navigate through the graph.


\subsection{Mandatory Criteria}

\begin{description}
  \item[M1] The app shall use the Bluetooth adapter of the smartphone to connect to a TI SimpleLink SensorTag device.
  \item[M2] The app shall track the position of a TI SimpleLink SensorTag device with up to 30m tolerance.
  \item[M3] The app shall visualize the sensors' data and its position using 3D/stereoscopy, more concrete the WebVR framework.
  \item[M4] The visualization mentioned in M3 shall be explorable by tilting the joystick of a bluetooth controller.
  \item[M5] The VR-World shall consists of at least two different rooms.
  \item[M7] The app shall display the stored data inside the VR-World.
  \item[M8] The data from the sensor shall have two different representations in the VR-World.
\end{description}

\subsection{Desired Criteria}

\begin{description}
  \item[A1] The app could visualize the sensors' data and its position using augmented reality.
  \item[A2] The VR-World could represent a whole corridor with more than two rooms.
  \item[A3] The app could give a time lapse of the data inside the VR-World.
\end{description}
